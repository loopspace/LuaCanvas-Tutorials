\documentclass[
  xhtml,%
  use filename%
]{internet}
%\usepackage{fancyvrb}
%\usepackage[scale=.8]{geometry}
\usepackage{tutorial}
\usepackage{hyperref}

\newtheorem{question}{Question}

\title{What's Your Angle?}

\begin{document}
\maketitle
\section{Introducing Logo}

\ \verb+Logo+ is a computer programming language built around giving instructions to a \emph{turtle} (can be either robotic or virtual).
You tell the turtle where to move, and as it moves then it draws a line to show its path.
The instructions are given as if telling the turtle where to go next.
So you might say ``Go forward \(50\) steps'' or ``Turn left \(30^\circ\)''.
Using these, and similar, commands you can get the turtle to draw quite intricate patterns.

There are a few other commands relating to the line that it draws.
You can change the colour and width of the line.
You can tell it to skip a section of the line.
In some implementations, you can tell it to \emph{fill} a shape that it has drawn (that isn't a feature of this version but you can find online \verb+Logo+ implementations to play with).

\section{Moving Your Turtle}

\begin{enumerate}
\item Go to \href{http://luacanvas.mathforge.org?project=Turtle}{http://luacanvas.mathforge.org?project=Turtle}
\item To make the turtle draw something, you give it the instructions inside the \verb+setup+ function.
It will then draw what you tell it when you run the code.

Your turtle is called just \verb+turtle+.
If you want to change its name, replace the lowercase \verb+turtle+ in the line \verb+turtle = Turtle()+ with your chosen name.
You can only use letters, and you can't use spaces (but capital letters are fine).
So \verb+Fred = Turtle()+ is okay.
If you change the name, you have to use that name exactly as you originally wrote it.
In this tutorial, we will assume that we haven't changed the name.

The commands are:
\begin{itemize}
\item To make the turtle go forwards, \verb+turtle:forward(30)+.

Note the colon, \verb+:+, between the \verb+turtle+ and \verb+forward+.
The number, in this case \(30\), tells it how much to go forwards.
The units are pixels (the tiny dots) on the screen.

The number can be negative, but \dots

\item To make the turtle go backwards, \verb+turtle:backward(30)+.

\item To turn the turtle to its left, \verb+turtle:left(30)+.

The key here is that the turn is from the point of view of the turtle.
It will turn to its left by \(30^\circ\).

The number can be negative, but \dots

\item To turn the turtle to its right, \verb+turtle:right(30)+.

\item To stop it drawing, use \verb+turtle:penUp()+.

Note the parentheses at the end.
This means that when the turtle moves it will not leave a trail behind it.

\item To resume drawing, use \verb+turtle:penDown()+.

\item To change the colour, use \verb+turtle:setPenColour(127,245,200)+ or \verb+turtle:setPenColour("blue")+.

For more on how to specify a colour, see the \href{Style.xhtml}{Programming with Style} tutorial in the \emph{Programming} section of the tutorials.

\item To change the width, use \verb+turtle:setPenWidth(3)+.

\end{itemize}

\item Put some commands in the code to see what happens.
For example, so that the start of your code looks like this:

\begin{verbatim}
function setup()
  turtle = Turtle()
  turtle:forward(100)
  turtle:right(60)
  turtle:forward(30)
  turtle:left(100)
  turtle:forward(50)
  protractor = Protractor()
  ruler = Ruler()
end
\end{verbatim}

\item When you run this code, the ``turtle'' will draw the lines you have told it to.
You will notice two other shapes on the screen (you may have noticed them in the code) which look a bit like a protractor and a ruler.
They are.
You can drag them around the screen and turn them to make measurements to plan your drawing.
They aren't labelled, but the marks are \(5^\circ\) on the protractor and \(5\) pixels on the ruler.
To move them, click near the middle of the shape and drag it.
To rotate them, click near the outside of the shape and drag it round.

\end{enumerate}

\section{A Regular Shape}

In a regular shape, all the sides have the same length and all the angles are the same.
You can easily make the turtle draw a regular shape by alternating a turn of a fixed angle with a forward of a fixed distance.
Written out in full, this would have a lot of repetition and would be tedious to change to make a new shape.
So we use a \emph{loop} to make it more concise.

To repeat something a set number of times, we put it in a \verb+for+ loop, like this:

\begin{verbatim}
for k=1,4 do
  turtle:forward(100)
  turtle:turn(50)
end
\end{verbatim}

This repeats the instructions a total of \(4\) times.
The start of your code should look like this:

\begin{verbatim}
function setup()
  turtle = Turtle()
  for k = 1,4 do
    turtle:forward(100)
    turtle:right(50)
  end
  protractor = Protractor()
  ruler = Ruler()
end
\end{verbatim}


This shape doesn't join up.
The first goal is to find an angle and a number of repetitions that do make a closed, simple shape.
``Closed'' means that the lines join up, ``simple'' means that the lines don't cross each other.

By changing the \verb+4+ in the line \verb+for k=1,4 do+ then we change the number of repetitions.
Changing the \verb+50+ changes the angle.
If your drawing is too big for the screen, change the \verb+100+ to something smaller.

Once you have found an angle and a number of repetitions that works, make sure that your number of repetitions is \emph{minimal}.
That means that if you reduce the number of repetitions further then the shape will no longer close up.
It will also mean that the turtle ends up at exactly the same place and in the same direction as it started.

\begin{question}
What is the relationship between the number of repetitions and the angle that the turtle turns each time?

Find some other angles and number of repetitions that also draw closed, simple shapes.

What is the general relationship?
\end{question}

\begin{question}
What happens if we no longer insist that the shape be \emph{simple}?
So we allow the lines to cross existing lines, but the turtle must end up back where it started and facing the same way as it started.
\end{question}

\section{Irregular Shapes}

In an irregular shape, we no longer insist that the angles and side lengths be the same.
Let us, though, insist that the shape be still a closed, simple shape.

Draw a closed, simple shape that only involves \emph{left} turns (i.e., left turns with a positive angle).

To draw your shape, start with a few instructions with turns and forwards, then run your code to see where your turtle ends up.
Keep adding some more instructions and running your code to see what happens.
You may find it useful to use the ruler and protractor, particularly when figuring out the last ``joining up'' line.

\begin{question}
What do you spot about the angles that you turn?

\end{question}

\emph{Hint: what do you get if you add them up?}

\begin{question}
Now include some right turns as well (but keep the shape simple: no intersections).
What do you notice about the angles now?
\end{question}

\emph{Hint: treat the left and right turns separately.}

\section{Spirals}

There are some additional drawing commands that can be used to make nice spirals.

\begin{itemize}
\item \verb+turtle:toPosition(20,30)+ moves the turtle to the coordinate \((20,30)\) (with origin in the centre of the screen).
If the turtle's pen is down, a line will be drawn as it moves.

\item \verb+turtle:toRelativePosition(20,30)+ moves the turtle to the
coordinate \((20,30)\) relative to its current position.
That is, it moves it \(20\) pixels across and \(30\) up.

\item \verb+turtle:toAngle(50)+ turns the turtle so that it makes an angle of \(50^\circ\) from the horizontal, measured anticlockwise.

\item \verb+turtle:toRelativeAngle(50)+ turns the turtle so that it makes an angle of \(50^\circ\) from the line joining it to its original position at the centre of the screen.
\end{itemize}

Here are some spirals to get you started.
Pick one and make some changes to the code to experiment with.

\begin{enumerate}

\item You can get a nice spiral with the following instructions:

\begin{verbatim}
turtle:penUp()
turtle:toPosition(10,0)
turtle:penDown()
for k = 1,100 do
  turtle:forward(10)
  turtle:toRelativeAngle(90)
end
\end{verbatim}

\item The \emph{Fibonacci Sequence} leads to a nice spiral:

\begin{verbatim}
local a,b = 5,5
turtle:right(135)
for k=1,10 do
  turtle:forward(a)
  turtle:left(90)
  a,b = b,a+b
end
\end{verbatim}

\item Spirals can go inwards as well as outwards:

\begin{verbatim}
local s = 100
for k=1,30 do
  turtle:forward(s)
  turtle:left(60)
  s = .9*s
end
\end{verbatim}
\end{enumerate}

\end{document}
