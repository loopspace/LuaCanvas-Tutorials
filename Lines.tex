\documentclass[
  xhtml,%
  use filename%
]{internet}

\usepackage{tutorial}
\usepackage{hyperref}

\title{Points and Lines}
\date{\today}
\begin{document}
\maketitle

\section{Getting Started}

\begin{enumerate}
\item Log on to the computer and launch a web browser
\item Navigate to \href{http://luacanvas.mathforge.org?project=Grid}{http://luacanvas.mathforge.org?project=Grid}
\item The big area in the centre of the screen is where you can type code (we call this the \emph{editor} region).
Clicking on the box labelled \verb+Execute+ will run your code.
When your code is running, clicking on the box labelled \verb+Edit+ will get you back to the editor.

\item There should be some code already in the editor.
Try running it, you should see a coordinate grid drawn on the screen.
\end{enumerate}

\section{Adding Coordinates}

As in the \href{Coordinates.xhtml}{Coordinates} tutorial, we will make considerable use of the \verb+point+ command, so let's make the same alias.
That is, put \verb+p=point+ in the \verb+setup+ function (i.e.\ after the line \verb+function setup()+ and before the first \verb+end+).

We'll also be adding some points, so we'll make a \emph{table} to put them in and draw them on the grid, just as in the \href{Coordinates.xhtml}{Coordinates} tutorial.

Your code should be (I've put in one point just to check that it is working):

\begin{verbatim}
function setup()
  g=Grid(20,5)
  p=point
  points={
    p(1,2)
  }
end

function draw()
  background(40,40,50)
  g:draw()
  fill("cyan")
  noStroke()
  for k,v in ipairs(points) do
    ellipse(v,.5)
  end
end
\end{verbatim}

We can now add points to our list and the program will display them on the screen.
Feel free to change the colour.
If you want to change the size of the circle, you can do that as well; you shouldn't make it any bigger than \(.5\).

\section{Points on a Line}

Let's pick a line and draw some points on that line.
We'll start with something simple to get the idea.
Let's start with the line of points whose \(y\)--coordinate is \(7\).
Remember that we write this line as ``\(y = 7\)''.

Think up some points on that line and put them in the \verb+points+ table.
So that line might read:

\begin{verbatim}
points={
  p(1,7), p(2,7), p(7,7)
}
\end{verbatim}

Note the commas \emph{between} the points.
It won't work if you forget them.

It's easy to see that these points all lie on the same line (assuming that you have typed in the coordinates of the points correctly).
Let's be more ambitious.
Let's now draw some points on the line ``\(y = x + 1\)''.
Remember that this means ``The line of points whose \(y\)--coordinate is one more than their \(x\)--coordinate.''.
As a start, the point \((2,3)\) lies on this line.
Replace your existing list by a list of points on this new line.

Your code should look something like this, but with different points.

\begin{verbatim}
function setup()
  g=Grid(20,5)
  p=point
  points={
    p(2,3), p(4,5), p(7,8)
  }
end

function draw()
  background(40,40,50)
  g:draw()
  fill("cyan")
  noStroke()
  for k,v in ipairs(points) do
    ellipse(v,.5)
  end
end
\end{verbatim}

If you are feeling confident, try doing this for the line \(2 y = 3 x + 2\).
Think first how we interpret this equation as a description of the line and then think of some points that lie on it.

\section{Drawing a Line}

It is possible that in your list you made a mistake; when you run the code then you should be able to tell if one of the points is off the line or not.
Try making a deliberate mistake to see the effect.

When making a deliberate mistake, you probably made the difference very big.
For example, the point \((2,-5)\) does not lie on the line \(y = x + 1\) and very obviously so.
But mistakes don't have to be big to be mistakes, and because small ones can be harder to spot, their effect can be larger than it should be.
For the line \(y = x + 1\), try the points:
\((-18,-16)\), \((2,3)\), \((4,5)\), and \((7,8)\).
Is it really obvious that \((-18,-16)\) is off the line?

One way to resolve the matter is to draw the actual line.
We need two points to draw a line (when drawing on paper it is good to have a third point as a ``check'' but the computer is more reliable and only needs two).
Unfortunately, while we are able to extend a line beyond the points, the computer will only draw the part of the line between the given points.
So we need to make sure that our points are wide apart.
It is allowed for the points to be ``off screen'', only the part on the screen will be drawn.

To draw the line \(y = x + 1\), we need a point off the bottom left of the screen.
The grid is set up so that the short direction goes from  \(-20\) to \(20\).
Therefore, the point \((-22,-21)\), which lies on the line, is definitely off the bottom left of the screen.
Similarly, \((21,22)\), which is also on the line, is definitely off the top right.
So we can use these two points to draw our line.
To do this, we use the command \verb+line(-22,-21,21,22)+.
If we put this \emph{after} the loop which draws our points then it will be easier to see whether or not our points are on the line.
We need to decide on its colour and width\begin{marginpar}The \emph{width} is in pixels and is not affected by the grid scale.\end{marginpar}.
To set the width, we write \verb+strokeWidth(2)+, and for the colour we write \verb+stroke("yellow")+ (we can change the \verb+2+ and \verb+yellow+ if we want).
These need to go before the \verb+line+ command but after the loop which draws our points.

Your code should now look something like the following:

\begin{verbatim}
function setup()
  g=Grid(20,5)
  p=point
  points={
    p(2,3), p(4,5), p(7,8), p(-18,-16)
  }
end

function draw()
  background(40,40,50)
  g:draw()
  fill("cyan")
  noStroke()
  for k,v in ipairs(points) do
    ellipse(v,.5)
  end
  strokeWidth(2)
  stroke("yellow")
  line(-22,-21,21,22)
end
\end{verbatim}

If you are feeling confident, try doing this for the line \(2 y = 3 x + 2\).
You might have to do a bit of thinking to find the right coordinates for the end points of your line.

\section{The Online Test}

Now that we've drawn the line, it is obvious that the point \((-18,-16)\) does not lie on the line.
But if we changed it to \((-18,-17.1)\) then it's not so obvious.
And if you think that it \emph{is} obvious, try \((-18,-17.01)\).

What we need is a \emph{test} to decide whether or not a given point is on our line.
The equation of the line is exactly that.
Remember that ``the line \(y = x + 1\)'' means ``the line of points whose \(y\)--coordinate is one more than their \(x\)--coordinate''.
So to test if a point is on the line we simply ask ``Is its \(y\)--coordinate equal to one more than its \(x\)--coordinate?''.
If the answer is yes, it is on the line; if no, it isn't.

So we want to translate the question ``Is its \(y\)--coordinate equal to one more than its \(x\)--coordinate?'' into something that a computer can understand.
Computers don't think\begin{marginpar}Computers don't think at all.  But it's a useful turn of phrase.\end{marginpar} in terms of questions.
They think in terms of \emph{tests}.
We run a test on a point and either the test is \emph{true} or \emph{false}.

The test we want is to see if the \(y\)--coordinate is equal to the \(x\)--coordinate plus one.
This is a slightly different meaning of ``equals'' than when we say ``let \(b = 3\)''.
In Lua, an equality \emph{test} is written with a double equals sign \verb+==+ (that's two equals signs next to each other).
So to test if the point \verb+v+ is on the line we would write:

\begin{verbatim}
if v.y == v.x + 1 then
\end{verbatim}

Note that to get at the \(y\)--coordinate of the point \verb+v+ we write \verb+v.y+ and to get at the \(x\)--coordinate we write \verb+v.x+. 

This is the start of the test, what we write next is what to do if the test is \emph{true}.
For example, we could change the colour of the dot:

\begin{verbatim}
if v.y == v.x + 1 then
  fill("yellow")
end
\end{verbatim}

The trouble with this is that once the condition is true for one point, the colour gets changed and stays changed.
We need to set it to something else if the test fails.
To do that we use the word \verb+else+.

\begin{verbatim}
if v.y == v.x + 1 then
  fill("yellow")
else
  fill("cyan")
end
\end{verbatim}

The meaning of this is as follows.
The test is checked, if it is true then the colour is set to yellow; if it is false then it is set to cyan.

We put this inside our loop before the ellipse command.
Thus our code will look something like the following.

\begin{verbatim}
function setup()
  g=Grid(20,5)
  p=point
  points={
    p(2,3), p(4,5), p(7,8), p(-18,-17.01)
  }
end

function draw()
  background(40,40,50)
  g:draw()
  fill("cyan")
  noStroke()
  for k,v in ipairs(points) do
    if v.y == v.x + 1 then
      fill("cyan")
    else
      fill("fuchsia")
    end
    ellipse(v,.5)
  end
  strokeWidth(2)
  stroke("yellow")
  line(-22,-21,21,22)
end
\end{verbatim}

You can try this for other lines as well.

\section{The Line Challenge}

It's a bit tedious to be typing in the set of points each time.
One way to make that a bit easier is to use the mouse to click on the grid and assemble the points that way.
When you click on the screen, the information about the click is passed to the program in a very special way.
To get the click coordinates we need to type a new block of code at the very end of what we have so far:

\begin{verbatim}
function touched(t)
  if t.state == ENDED then
    table.insert(points,p(t))
  end
end
\end{verbatim}

This won't work.
The problem is that the information about the mouse click uses coordinates relative to the screen and measured in pixels.
It doesn't know about our grid.
Fortunately, we do and our grid knows how to convert the coordinates.
So what we actually want is:

\begin{verbatim}
function touched(t)
  if t.state == ENDED then
    table.insert(points,g:invtransform(t))
  end
end
\end{verbatim}

This rounds to the nearest grid square.
If you want to allow more precision, replace \verb+g:invtransform(t)+ with \verb+g:invtransform(t,1)+.

Our complete code is now:

\begin{verbatim}
function setup()
  g=Grid(20,5)
  p=point
  points={
    p(2,3), p(4,5), p(7,8), p(-18,-17.01)
  }
end

function draw()
  background(40,40,50)
  g:draw()
  fill("cyan")
  noStroke()
  for k,v in ipairs(points) do
    if v.y == v.x + 1 then
      fill("cyan")
    else
      fill("fuchsia")
    end
    ellipse(v,.5)
  end
  strokeWidth(2)
  stroke("yellow")
  line(-22,-21,21,22)
end

function touched(t)
  if t.state == ENDED then
    table.insert(points,g:invtransform(t,1))
  end
end
\end{verbatim}

What you can do now is make a challenge for a friend.
Choose a new equation of a line
(don't make it too complicated first time round).
Clear out the points from the \verb+points+ table, remove the code that actually draws the line, and change the test so that it matches your line not \(y = x + 1\) (remember to use the double equals sign).

So your code should look like the following, only with a different test:

\begin{verbatim}
function setup()
  g=Grid(20,5)
  p=point
  points={}
end

function draw()
  background(40,40,50)
  g:draw()
  fill("cyan")
  noStroke()
  for k,v in ipairs(points) do
    if v.y == v.x + 1 then
      fill("cyan")
    else
      fill("fuchsia")
    end
    ellipse(v,.5)
  end
end

function touched(t)
  if t.state == ENDED then
    table.insert(points,g:invtransform(t))
  end
end
\end{verbatim}

Now start the code and challenge a friend to discover the equation of your line by finding points on it.

\section{Some Extras}

Here are some ideas on how to take things further.

\begin{itemize}
\item In the challenge game, have a button that if pressed draws the line.
To create such a button, put the following in the \verb+setup+ function:

\begin{verbatim}
parameter.boolean("Draw line","doline",false)
\end{verbatim}

This creates a \emph{boolean} called \verb+doline+.
A boolean is a special variable that is either \emph{true} or \emph{false}.
It can be used in a test, so you can write:

\begin{verbatim}
if doline then
  ...
end
\end{verbatim}

The code in the middle is used if \verb+doline+ is true, but is ignored if not.

\item You might also want to be able to clear the table of points.
You can create a button that does this.
In the \verb+setup+ function write:

\begin{verbatim}
parameter.action("Clear points",function() points = {} end) 
\end{verbatim}

\item If you want to draw more than one line, you could make a table of lines.
For each line, you might want some points on it, a colour, the end points for drawing it, and even a test function for whether a point is on the line.
You might have something like:

\begin{verbatim}
lines = {
  {
    colour = colour("yellow"),
    points = {p(1,2), p(3,4), p(7,8)},
    ends = {p(-22,-21),p(21,22)},
    test = function(c) return c.y == c.x + 1 end
  },
  {
    colour = colour("cyan"),
    points = {p(2,4), p(4,7), p(0,1)},
    ends = {p(-20,-29),p(20,31)},
    test = function(c) return 2*c.y == 3*c.x + 2 end
  },
}
\end{verbatim}

Then in the \verb+draw+ function, we need to iterate over the lines first.

\begin{verbatim}
for l,u in ipairs(lines) do
  noStroke()
  fill(u.colour)
  for k,v in ipairs(u.points) do
    ellipse(v,.5)
  end
  stroke(u.colour)
  strokeWidth(2)
  line(u.ends[1],u.ends[2])
end
\end{verbatim}

For testing points from the mouse, we could do:

\begin{verbatim}
for k,v in ipairs(points) do
  fill("green")
  for l,u in ipairs(lines) do
    if u.test(v) then
      fill(u.colour)
    end
  end
  ellipse(v,.5)
end
\end{verbatim}
\end{itemize}

Putting all that together, our full code looks like:

\begin{verbatim}
function setup()
  g=Grid(20,5)
  parameter.action("Clear points",function() points = {} end)
  parameter.boolean("Draw lines","doline",false)
  p=point
  lines = {
    {
      colour = colour("yellow"),
      points = {p(1,2), p(3,4), p(7,8)},
    	ends = {p(-22,-21),p(21,22)},
      test = function(c) return c.y == c.x + 1 end
    },
    {
      colour = colour("cyan"),
      points = {p(2,4), p(4,7), p(0,1)},
      ends = {p(-20,-29),p(20,31)},
      test = function(c) return 2*c.y == 3*c.x + 2 end
    },
	}
  points={}
end

function draw()
  background(40,40,50)
  g:draw()
  for l,u in ipairs(lines) do
    noStroke()
    fill(u.colour)
    for k,v in ipairs(u.points) do
      ellipse(v,.5)
    end
    if doline then
      stroke(u.colour)
      strokeWidth(2)
      line(u.ends[1],u.ends[2])
    end
  end
  noStroke()
  for k,v in ipairs(points) do
    fill("green")
    for l,u in ipairs(lines) do
      if u.test(v) then
        fill(u.colour)
      end
    end
    ellipse(v,.5)
  end
end

function touched(t)
  if t.state == ENDED then
    table.insert(points,g:invtransform(t))
  end
end
\end{verbatim}

\end{document}
