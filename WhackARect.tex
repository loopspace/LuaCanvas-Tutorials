\documentclass[
  xhtml,%
  use filename%
]{internet}

\usepackage{tutorial}
\usepackage{hyperref}

\title{Whack a Rect}
\date{\today}
\begin{document}
\maketitle

\section{Getting Started}

\begin{enumerate}
\item Log on to a computer and launch a web browser (Chrome or IE should be installed, Chrome is marginally better).
\item In the location bar, enter \href{http://luacanvas.mathforge.org}{http://luacanvas.mathforge.org}
\item You should now have a webpage with some code in it, looking something like the following:

\begin{verbatim}
function setup()
  print("hello world")
end

function draw()
  background(40,40,50)
  fill(150,200,30)
  stroke(200,30,150)
  strokeWidth(10)
  rect(20,20,100,100)
end
\end{verbatim}

Click on the button marked \verb+Execute+ in the bottom left of your screen to run the code.
Click on \verb+Edit+ to get back to the code editor.
\end{enumerate}

\section{What's Happening}

The code works as follows.

\begin{itemize}
\item Everything between \verb+function setup()+ and the first \verb+end+ is run once.
The \verb+print("hello world")+ is what puts the words \verb+hello world+ in the box on the left of the screen.
You can change this message if you like, but you need the double inverted commas and the brackets as they are.

\item Then everything between \verb+function draw()+ and the second \verb+end+ is run.
This is run in a constant loop, several times a second.
You can't see this at the moment because it is doing the same thing each time.

\begin{itemize}
\item \verb+backgroud(40,40,50)+ This blanks the \emph{canvas} by filling it with a solid colour.
Try changing the numbers and see what happens (you must have three numbers).

\item The lines

\begin{verbatim}
  fill(150,200,30)
  stroke(200,30,150)
  strokeWidth(10)
\end{verbatim}

Don't \emph{do} anything but they set up certain properties of how things will be drawn on the screen.
Shapes can be \emph{filled} and/or \emph{stroked} (which means that their outline is drawn).
These commands set the colours for the fill and stroke, and the width of the stroke.

\item \verb+rect(20,20,100,100)+ Finally, this draws a rectangle on the screen with lower left corner at \verb+(20,20)+ (ie \(20\) pixels in from the lower left corner of the screen and \(20\) pixels up).
It is \(100\) pixels in width and \(100\) pixels in height.
\end{itemize}
\end{itemize}

\section{Variables}

One of the basic pieces of any computer program is the use of \emph{variables} to store information which can be later retrieved and used.

\subsection{Setting and Using Variables}

The language we're using here is Lua and in Lua the syntax to set a variable is simply \verb+name = value+.
For now, we'll say that a variable name is a string of letters, such as \verb+message+ or \verb+x+ (some other characters can be used, but definitely no spaces).
The value can be quite varied, we'll start with a number.

In the \verb+setup+ function (ie between \verb+function setup()+ and the first \verb+end+) start a new line (press \verb+return+ at the end of an existing line) and type \verb+x = 20+.
This sets the value of \verb+x+ to \verb+20+.
So later on in the program, we can use \verb+x+ in place of \verb+20+.
Let's do that by changing the first \verb+20+ in the \verb+rect+ line to \verb+x+ so that it reads \verb+rect(x,20,100,100)+.
Our program should now look like the following:

\begin{verbatim}
function setup()
  print("hello world")
  x = 20
end

function draw()
  background(40,40,50)
  fill(150,200,30)
  stroke(200,30,150)
  strokeWidth(10)
  rect(x,20,100,100)
end
\end{verbatim}


If we run the program we shouldn't spot any difference.
But if we go back and change the \verb+x = 20+ to, say, \verb+x = 100+ then we should see a change.

Do the same for the \(y\)--coordinate and the size of the rectangle.
Use \verb+y+ for the \(y\)--coordinate and \verb+s+ for the size.
We'll keep our rectangle square for now, so replace \emph{both} \(100\)s by \verb+s+.
Try changing \verb+x+, \verb+y+, and \verb+s+ to see what you get.
How big can they get without going off the screen?

Your program should now look something like the following (possibly with different values for the variables).

\begin{verbatim}
function setup()
  print("hello world")
  x = 20
  y = 20
  s = 100
end

function draw()
  background(40,40,50)
  fill(150,200,30)
  stroke(200,30,150)
  strokeWidth(10)
  rect(x,y,s,s)
end
\end{verbatim}

\subsection{Random Values}

Now let's make it a bit more interesting by getting the computer to pick those values.
We'll get it to pick them at random.
To do this, we need a function that returns a random number in some range.
Lua has one which is called \verb+math.random+.
If we give it a single value, it will return a random integer between \(1\) and that value.
We want to be sure that our square stays on the screen so we need to know how big the screen is.
Fortunately, we don't have to measure it: the program knows the size and stores it in the variables \verb+WIDTH+ and \verb+HEIGHT+ (note: both all uppercase).
So replace the line in \verb+setup+ that sets \verb+x+ with \verb+x = math.random(WIDTH-100)+ (the \(-100\) is because this will be the lower left corner of the square).
Do the same for the \verb+y+ variable, only use \verb+HEIGHT+ instead of \verb+WIDTH+.

Your code should now look like this:

\begin{verbatim}
function setup()
  print("hello world")
  x = math.random(WIDTH-100)
  y = math.random(HEIGHT-100)
  s = 100
end

function draw()
  background(40,40,50)
  fill(150,200,30)
  stroke(200,30,150)
  strokeWidth(10)
  rect(x,y,s,s)
end
\end{verbatim}

Each time you rerun your code, the square should change position (you can rerun your code quickly by clicking \verb+Restart+).

\subsection{Changing Values}

The computer can also change the value of a variable.
Inside the \verb+draw+ function, (after the \verb+rect+ line) put \(s = s - 1\).
This tells the computer to set the \verb+s+ variable to whatever it currently is minus \(1\).
Your code should now look like this:

\begin{verbatim}
function setup()
  print("hello world")
  x = math.random(WIDTH-100)
  y = math.random(HEIGHT-100)
  s = 100
end

function draw()
  background(40,40,50)
  fill(150,200,30)
  stroke(200,30,150)
  strokeWidth(10)
  rect(x,y,s,s)
  s = s - 1
end
\end{verbatim}

Now when you run your code you will see your square get smaller (and then bigger again!  Can you explain why?).

\subsection{Conditionals}

A key component of any program is the use of \emph{conditionals}.
These say ``If a condition is true, do something'', meaning that we do an action only under certain conditions.
Let's say that when our square gets to zero size, we reset it.
So after the line \verb+s = s - 1+ write:

\begin{verbatim}
if s == 0 then
  x = math.random(WIDTH-100)
  y = math.random(HEIGHT-100)
  s = 100
end
\end{verbatim}

(You can use the standard copy-and-paste tools to copy the middle bit from earlier in the code.)

Your code should now look like the following.

\begin{verbatim}
function setup()
  print("hello world")
  x = math.random(WIDTH-100)
  y = math.random(HEIGHT-100)
  s = 100
end

function draw()
  background(40,40,50)
  fill(150,200,30)
  stroke(200,30,150)
  strokeWidth(10)
  rect(x,y,s,s)
  s = s - 1
  if s == 0 then
    x = math.random(WIDTH-100)
    y = math.random(HEIGHT-100)
    s = 100
  end
end
\end{verbatim}


Now you should see that the square vanishes and reappears somewhere else.

(You may want to change the \emph{style} parameters; feel free to do so.)

\section{Interaction}

So far so good.
Let's make a game out of this.
To do that, we need to make it interactive.

If you click on the screen, nothing happens.
But the computer notices that click and can do something with it if we tell it to.
To make it notice the click, we need to define a new \emph{function}.
So at the very end of your code, type:

\begin{verbatim}
function touched(t)

end
\end{verbatim}

This will get called when you click on the screen (actually it gets called slightly more often than that).
The \verb+(t)+ part means that when it is called, a variable \verb+t+ is set which contains lots of information about the click.
The key information that we want are the \emph{state} and \emph{position} of the click.
We use the \emph{state} to ensure that our code only runs when the user clicks, and the \emph{position} to see if the user has clicked inside the rectangle.
Our full function is (anything after \verb+--+ is a \emph{comment} and is ignored by the computer; you can ignore it as well -- i.e., you don't have to type it in, and lines with just a comment on can also be skipped):

\begin{verbatim}
function touched(t)
    if t.state == BEGAN then -- detect whether we're clicking
                         -- or doing something else with the mouse
        if t.x > x -- are we to the right of the left side of the rectangle?
            and t.x < x + s -- are we also to the left of the right side?
            and t.y > y -- are we also above the lower side?
            and t.y < y + s -- are we also below the upper side?
            then -- it is okay to split complicated instructions
                    -- on several lines like this
              -- make a new rectangle
              x = math.random(WIDTH-100)
              y = math.random(HEIGHT-100)
              s = 100
        end -- end the position test
    end -- end the state test
end -- end the function
\end{verbatim}

Now when you run it, you can try to click in a square.
If you do so, it will vanish and return in a new place.

Your full code should look something like the following:

\begin{verbatim}
function setup()
  print("hello world")
  x = math.random(WIDTH-100)
  y = math.random(HEIGHT-100)
  s = 100
end

function draw()
  background(40,40,50)
  fill(150,200,30)
  stroke(200,30,150)
  strokeWidth(10)
  rect(x,y,s,s)
  s = s - 1
  if s == 0 then
    x = math.random(WIDTH-100)
    y = math.random(HEIGHT-100)
    s = 100
  end
end

function touched(t)
    if t.state == "BEGAN" then
        if t.x > x
            and t.x < x + s
            and t.y > y
            and t.y < y + s
            then
              x = math.random(WIDTH-100)
              y = math.random(HEIGHT-100)
              s = 100
        end
    end
end
\end{verbatim}


\section{Going Further}

This isn't quite a finished game.
We should keep score as well.
You have most of the pieces for doing this.
The only missing piece is to be able to display the contents of a variable.
In \verb+setup+, put \verb+score = 0+ to initialise the score.
At the end of \verb+draw+, put \verb+text("Score " .. score, 20,20)+ to display the current score.

Exactly how the score is added up is up to you.
Maybe you just want to count the number of successes.
Maybe you want to subtract one for every square that disappears without being hit.
Maybe you want the score to depend on how big the square is when it is hit.
It's up to you!
                


\end{document}
