\documentclass[
  xhtml,%
  use filename%
]{internet}

\usepackage{tutorial}
\usepackage{hyperref}

\title{A Christmas Card}
\date{\today}
\begin{document}
\maketitle

\section{Introduction}

There are several built-in shapes that you can use to make a scene on the screen, but sometimes you need a little more flexibility.
In this tutorial, we'll introduce \verb+path+s and see how we can use them to make almost any shape.

\section{The Basics}

A \verb+path+ is what it sounds like: a path on the screen.
It is built up out of pieces like lines and arcs.
To make a path, you first give it a name.
Then you add pieces to it.

For example, in the following code we define a simple shape in the \verb+setup+ function.

\begin{verbatim}
function setup()
  print("hello world")
  shape = path()
  shape:moveTo(200,200)
  shape:lineTo(300,200)
  shape:lineTo(200,300)
end

function draw()
  background(40,40,50)
  fill(150,200,30)
  stroke(200,30,150)
  strokeWidth(10)
  rect(20,20,100,100)
  shape:fill()
  shape:draw()
end
\end{verbatim}

Defining the shape is the lines:

\begin{verbatim}
shape = path()
shape:moveTo(200,200)
shape:lineTo(300,200)
shape:lineTo(200,300)
\end{verbatim}

The first line names our path \verb+shape+.
The second line \emph{moves} our path to the point \((200,200)\).
This tells our path where to start.
The third line then adds a line to the path, starting at its current point (which is \((200,200)\)) and ending at \((300,200)\).
Every time we add a piece to the path, it starts where the last one ended.
Then the last line adds another line from \((300,200)\) to \((200,300)\).

This sets up the shape.
To put something on the screen, we have the lines in the \verb+draw+ function:

\begin{verbatim}
shape:fill()
shape:draw()
\end{verbatim}

The first line \emph{fills} the path with the current \verb+fill+ colour.
As our path is not closed, it joins it up to close it before filling it.
The second line draws the path with the current \verb+stroke+ colour and \verb+strokeWidth+.
(The path isn't closed when it is drawn.)

\section{Drawing a Tree}

We can draw a Christmas tree using paths.
Let's start with naming our path: put \verb+tree = path()+ in the \verb+setup+ function.
We'll define it so that the top of the tree is at the point \verb+(0,0)+ and then use \emph{transformations} to move it to where we want it.
To make our tree start at \verb+(0,0)+, we write \verb+tree:moveTo(0,0)+.

We want to start with a nice curve.
We can do that with an arc.
To add an arc, we need to fix a centre, a radius, and starting and ending angles.
So to draw a quarter circle starting at the top and sweeping down, we could use:

\begin{verbatim}
tree:arc(-100,0,100,0,-90,true)
\end{verbatim}

This arc is drawn from a circle centred at \verb+(-100,0)+ and has radius \verb+100+.

We'll want another arc the other side.

\begin{verbatim}
tree:arc(100,0,100,-90,-180,true)
\end{verbatim}

We'll just be filling the shape so don't need to worry about joining up the arcs.

To draw the shape, we need to move it and then fill it.
In the \verb+draw+ function, we write:

\begin{verbatim}
tree:use({transformShape = true, fill})
\end{verbatim}

Putting all that together, we have:

\begin{verbatim}
function setup()
  print("hello world")
  tree = path()
  tree:moveTo(0,0)
  tree:arc(-100,0,100,0,-90,true)
  tree:arc(100,0,100,-90,-180,true)
end

function draw()
  background(40,40,50)
  fill(50,200,30)
  translate(WIDTH/2,HEIGHT-50)
  tree:use({transformShape = true, fill = true})
end
\end{verbatim}

We can add another segment of the tree with more arcs:

\begin{verbatim}
tree:arc(-200,0,200,0,-90,true)
tree:arc(200,0,200,-90,-180,true)
\end{verbatim}

We can add a third segment by changing the \verb+200+s to \verb+300+s, but that doesn't look quite right.
Better is to use a smaller arc slightly lower down.

\begin{verbatim}
tree:arc(-250,-50,250,0,-90,true)
tree:arc(250,-50,250,-90,-180,true)
\end{verbatim}

Our code at this point looks like the following:

\begin{verbatim}
function setup()
  print("hello world")
  tree = path()
  tree:moveTo(0,0)
  tree:arc(-100,0,100,0,-90,true)
  tree:arc(100,0,100,-90,-180,true)
  tree:arc(-200,0,200,0,-90,true)
  tree:arc(200,0,200,-90,-180,true)
  tree:arc(-250,-50,250,0,-90,true)
  tree:arc(250,-50,250,-90,-180,true)
end

function draw()
  background(40,40,50)
  fill(50,200,30)
  stroke(0,0,0)
  strokeWidth(5)
  translate(WIDTH/2,HEIGHT-50)
  tree:use({transformShape = true, fill = true})
end
\end{verbatim}

\section{The Trunk and Bucket}

The trunk and the bucket will be a new paths, one for each.
One reason for this is that we will fill them with different colours.
So in \verb+setup+ we need paths \verb+trunk = path()+ and \verb+bucket = path()+.
We'll actually draw the trunk first so it can overlap slightly with the tree and bucket.
So it just needs to be a rectangle.
The centre of the bottom of the tree is at \verb+(0,-300)+, so we start just left of there:

\begin{verbatim}
trunk:moveTo(-10,-300)
trunk:lineTo(-10,-350)
trunk:lineTo(10,-350)
trunk:lineTo(10,-300)
\end{verbatim}

When filling each path, we can specify the colour as part of its options.
This saves us having to remember which colour goes with which piece.
So in the \verb+draw+ function, we can write:

\begin{verbatim}
function setup()
  print("hello world")
  tree = path()
  tree:moveTo(0,0)
  tree:arc(-100,0,100,0,-90,true)
  tree:arc(100,0,100,-90,-180,true)
  tree:arc(-200,0,200,0,-90,true)
  tree:arc(200,0,200,-90,-180,true)
  tree:arc(-250,-50,250,0,-90,true)
  tree:arc(250,-50,250,-90,-180,true)
  trunk = path()
  trunk:moveTo(-10,-300)
  trunk:lineTo(-10,-350)
  trunk:lineTo(10,-350)
  trunk:lineTo(10,-300)
end

function draw()
  background(40,40,50)
  fill(50,200,30)
  stroke(0,0,0)
  strokeWidth(5)
  translate(WIDTH/2,HEIGHT-50)
  trunk:use({transformShape = true, fill = colour("brown")})
  tree:use({transformShape = true, fill = colour(50,200,30)})
end
\end{verbatim}

Of course, we can use other colours.
See the \href{Style.xhtml}{Style} tutorial for more details.

Now for the bucket.
To give it a more 3D look, we'll curve the top and bottom.
So in \verb+setup+:

\begin{verbatim}
bucket = path()
bucket:moveTo(-30,-340)
bucket:lineTo(-20,-400)
bucket:curveTo(-10,-410,10,-410,20,-400)
bucket:lineTo(30,-340)
bucket:curveTo(20,-350,-20,-350,-30,-340)
\end{verbatim}

And in \verb+draw+:

\begin{verbatim}
trunk:use({transformShape = true, fill = colour("brown")})
bucket:use({transformShape = true, fill= colour("red")})
tree:use({transformShape = true, fill = colour(50,200,30)})
\end{verbatim}

Our full code looks like this:

\begin{verbatim}
function setup()
  print("Merry Christmas!")
  tree = path()
  tree:moveTo(0,0)
  tree:arc(-100,0,100,0,-90,true)
  tree:arc(100,0,100,-90,-180,true)
  tree:arc(-200,0,200,0,-90,true)
  tree:arc(200,0,200,-90,-180,true)
  tree:arc(-250,-50,250,0,-90,true)
  tree:arc(250,-50,250,-90,-180,true)
  trunk = path()
  trunk:moveTo(-10,-300)
  trunk:lineTo(-10,-350)
  trunk:lineTo(10,-350)
  trunk:lineTo(10,-300)
  bucket = path()
  bucket:moveTo(-30,-340)
  bucket:lineTo(-20,-400)
  bucket:curveTo(-10,-410,10,-410,20,-400)
  bucket:lineTo(30,-340)
  bucket:curveTo(20,-350,-20,-350,-30,-340)
end

function draw()
  background(40,40,50)
  translate(WIDTH/2,HEIGHT-50)
  trunk:use({transformShape = true, fill = colour("brown")})
  bucket:use({transformShape = true, fill= colour("red")})
  tree:use({transformShape = true, fill = colour(50,200,30)})
end
\end{verbatim}

\section{Decorations}

One we have our basic tree, we can decorate it how we like.
We could use \verb+ellipse+ or \verb+circle+ to make baubles, and \verb+rect+ to make presents.
Or we could define new paths to make more complicated objects, like a star for the top.

If we're feeling very ambitious, we could even animate the decorations so that they move around the tree.

\end{document}
