\documentclass[
  xhtml,%
  use filename%
]{internet}

\loadcss{css/mathml}
\loadcss{css/numberedeqns}
\loadcss{css/article}
\loadfullcss{https://fonts.googleapis.com/css?family=Raleway:300}
\loadcss[all and (max-width: 767px)]{css/small}
\loadcss[all and (min-width: 768px) and (max-width: 1024px)]{css/medium}
\loadcss[all and (min-width: 1025px)]{css/large}
\loadjs{js/setwidth}
\loadfulljs{http://cdn.mathjax.org/mathjax/latest/MathJax.js?config=MML_HTMLorMML}

\usepackage{hyperref}

\title{Program with Style}
\date{\today}
\begin{document}
\maketitle

\section{Introduction}

With LuaCanvas, one can draw lots of different things on the screen and in different styles.
This tutorial will take you through the various things that can be drawn and how they can be styled.
Make sure you have \href{http://loopspace.github.io/LuaCanvas}{http://loopspace.github.io/LuaCanvas} open in a tab on your browser.

\section{What You Can Draw}

The default program is:

\begin{verbatim}
function setup()
  print("hello world")
end

function draw()
  background(40,40,50)
  fill(150,200,30)
  stroke(200,30,150)
  strokeWidth(10)
  rect(20,20,100,100)
end
\end{verbatim}

The key drawing line here is the second-to-last one: \verb+rect(20,20,100,100)+.
This draws a rectangle with lower-left corner at \((20,20)\) with width \(100\) and height \(100\).
You can experiment with changing those numbers to see what you get.
Try working out how wide and high your drawing space is by altering the numbers (you will get different answers depending on how big your browser window is).

You can also draw ellipses with the command \verb+ellipse+.
Change the \verb+rect+ to \verb+ellipse+ to see what happens.
Again, try changing the numbers.

As well as the rectangle and ellipse shapes, you can draw lines.
The command for drawing a line is:

\begin{verbatim}
line(sx,sy,tx,ty)
\end{verbatim}

where \((sx,sy)\) are the coordinates of the start of the line and \((tx,ty)\) are the coordinates of the end.
For example

\begin{verbatim}
line(30,40,200,250)
\end{verbatim}

draws a line from \((30,40)\) to \((200,250)\).

Using a combination of \verb+rect+, \verb+ellipse+, and \verb+line+ you can draw lots of things on the screen.

There are currently two more commands that draw on the canvas: \verb+arc+ and \verb+bezier+.
The \verb+arc+ command draws an arc with a radius, starting angle, and ending angle.
The \verb+bezier+ command draws a special type of curve called a \emph{cubic b\'ezier} curve.
It takes four sets of coordinates: it starts at the first coordinate, leaves in the direction of the second, and it ends at the fourth coordinate arriving from the direction of the third coordinate.
For example:

\begin{verbatim}
arc(300,50,20,45,135)
bezier(200,200,200,300,300,200,300,300)
\end{verbatim}

\section{Styling}

\end{document}